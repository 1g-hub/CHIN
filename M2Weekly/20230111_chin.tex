%\documentstyle[epsf,twocolumn]{jarticle}       %LaTeX2.09�d�l
%\documentclass[twocolumn]{jarticle} 
\documentclass[draftclsnofoot,onecolumn]{jarticle}
\usepackage{float} %设置图片浮动位置的宏包
\usepackage{subfigure} %插入多图时用子图显示的宏包
\usepackage{url}
%%%%%%%%%%%%%%%%%%%%%%%%%%%%%%%%%%%%%%%%%%%%%%%%%%%%%%%%%%%%%%
%%
%%  n�{ �o�[�W����
%%
%%%%%%%%%%%%%%%%%%%%%%%%%%%%%%%%%%%%%%%%%%%%%%%%%%%%%%%%%%%%%%%%
\setlength{\topmargin}{-45pt}
%\setlength{\oddsidemargin}{0cm} 
\setlength{\oddsidemargin}{-7.5mm}
%\setlength{\evensidemargin}{0cm} 
\setlength{\textheight}{24.1cm}
%setlength{\textheight}{25cm} 
\setlength{\textwidth}{17.4cm}
%\setlength{\textwidth}{172mm} 
\setlength{\columnsep}{11mm}


%�y�߂������邲�Ƃ�(1.1)(1.2) �c(2.1)(2.2)�Ɛ����ԍ����‚����Ƃ��z
%\makeatletter
%\renewcommand{\theequation}{%
%\thesection.\arabic{equation}} %\@addtoreset{equation}{section}
%\makeatother

%\renewcommand{\arraystretch}{0.95} �sT�̐ݒ�

%%%%%%%%%%%%%%%%%%%%%%%%%%%%%%%%%%%%%%%%%%%%%%%%%%%%%%%%
\usepackage[dvipdfmx]{graphicx}   %pLaTeX2e�d�l(�v\documentstyle ->\documentclass)
%%%%%%%%%%%%%%%%%%%%%%%%%%%%%%%%%%%%%%%%%%%%%%%%%%%%%%%%

\begin{document}

\onecolumn
\noindent

\hspace{1em}

\ \ 令和 5 年 1 月 11 日 ( 水 ) ゼミ資料 
\hfill
\ \ M2 陳偉斉


\vspace{2mm}

\hrule

\begin{center}
{\Large \bf 進捗報告}
\end{center}


\hrule
\vspace{3mm}

\section{今週やったこと}
	\begin{itemize}
	  \item baseline 実験\vspace{-0.5em}
	\end{itemize}

\section{実験内容}
SVM, LSTM, bert-base-chinese モデルに対して実験しました.

\section{データセット}
「問題文 + ヒント + 答え」,「正解」で構成されるの実験データ 72937 件集まりました.
そして Levenshteinを利用し,「不正解だけと似てる漢字(画と SUB 漢字両方も Levenshtein 距離が1)」の条件でデータ数を 218811 件に増加しました.
bert-base-chinese モデルのみに対して,「明らかに不正解 ( 画と SUB 漢字両方も Levenshtein 距離が 20 以上)」と「ヒントなし」の条件を設定し実験をしました.他のモデルはまだ実験中です.

データセットの構成は表 1 のように示します.

\begin{table}[h]
\centering
\caption{データセットの構成}
\label{a}
\centering
\resizebox{\linewidth}{!}{
\begin{tabular}{|c|c|c|c|}
\hline 条件 & 訓練データ & テストデータ & 合計\\
\hline 似てる不正解 & 175048 & 43763 & 218811\\
\hline 明らかに不正解 & 175048 & 43763 & 218811\\
\hline ヒントなし & 175048 & 43763 & 218811\\
\hline
\end{tabular}
}
\end{table}

\section{実験結果}

実験結果は表 2 ,表 3 に示します.

\begin{table}[h]
\centering
\caption{各モデルの実験結果}
\label{b}
\centering
\resizebox{\linewidth}{!}{
\begin{tabular}{|c|c|c|c|c|}
\hline 結果 & SVM(CountVectorizer) & SVM(TfidfVectorizer) & LSTM(bidirectional) & bert-base-chinese\\
\hline 訓練誤差 &  &  & 0.21 & 0.21\\
\hline 訓練精度 & 0.67 & 0.68 & 0.90 & 0.90\\
\hline テスト精度 & 0.66 & 0.56 & 0.86 & 0.78\\
\hline
\end{tabular}
}
\end{table}

その中に最も表現がいいのは LSTM モデルですが, Bert の訓練は時間かかりますので, 毎回 20 Epoch を設定しました(コスト 6 時間).それに反して, LSTM は 50 Epoch dでした. 故に Bert モデルは訓練不足の可能性もあります. 

\begin{table}[h]
\centering
\caption{Bertと違うデータセットの実験結果}
\label{c}
\centering
\resizebox{\linewidth}{!}{
\begin{tabular}{|c|c|c|c|}
\hline 条件 & 似てる不正解 & 明らかに不正解 & ヒントなし\\
\hline 訓練誤差 & 0.21 & 0.16 & 0.40\\
\hline 訓練精度 & 0.90 & 0.94 & 0.85\\
\hline テスト精度 & 0.78 & 0.76 & 0.78\\
\hline
\end{tabular}
}
\end{table}
結果として,人類の判断に影響する「漢字の形」と「ヒント」が bert-base-chinese に与える影響は僅かですが,今回の実験は全部漢字を最小単位として扱うため,画と SUB 漢字は今回の実験に導入されてません.次に導入します.
\section{来週目標}
 \begin{itemize}
	\item Bert を 100 epoch に増加し実験する
	\item 不正解データを増加し実験する
	\item 画と SUB 漢字を導入する(データの中に分けるか,分散表現を生成するか)

	\end{itemize}


\bibliography{weekly}
\bibliographystyle{junsrt}

\end{document}


